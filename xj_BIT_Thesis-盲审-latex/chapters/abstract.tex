%%==================================================
%% abstract.tex for BIT Master Thesis
%% modified by yang yating
%% version: 0.1
%% last update: Dec 25th, 2016
%%==================================================

\begin{abstract}
事件抽取作为信息抽取领域的关键组成部分,其提供的结构化语义信息广泛应用在事实核查、信息检索、智能问答、推荐系统等多样化自然语言场景中。此外,以事件抽取为基础构建的事件知识图谱同样具备了广阔的应用前景。随着深度学习技术的快速发展,现有方法利用特定神经网络架构研究事件抽取的不同阶段任务。尽管这些方法取得了优异的性能进步,仍存在如下的共性数据挑战:1)不同事件检测方法存在的数据不平衡依赖,限制了其识别事件的能力,进一步降低了整体检测性能;2)不同事件要素抽取方法存在的实体类型过度依赖,阻碍了其学习实体类型以外的更多语义信息,影响了事件要素类型的正确预测;3)当事件要素抽取任务扩展到多级别文本时,不同事件要素抽取方法存在的内在局限和跨事件信息限制,阻碍了其从不同级别文本中有效构建跨事件依赖关系,限制了多事件的要素抽取能力。为此,本文围绕事件抽取中的共性数据依赖展开研究,主要研究内容和创新成果如下:

(1)提出基于分类器自适应知识蒸馏的数据不平衡依赖消解方法。针对事件检测中存在的数据不平衡依赖,首先研究其影响事件检测性能的方式,并相应定义和引入句子级别识别信息,以主动消解数据不平衡依赖导致的性能损失。在此基础上,提出一种基于分类器自适应知识蒸馏的事件检测方法。具体地,该方法在事件识别增强网络中引入根据训练实例标签转化的句子级别识别信息,并将其作为输入的一部分参与训练。进一步,设计额外的训练任务,通过事件识别增强网络的分类器引导事件检测网络从原始文本中自动学习句子级别识别信息,从而实现事件检测性能的提升。实验评估表明,提出的方法可以通用有效地消解数据不平衡依赖对于不同事件检测模型的性能影响,并展现自主适应不同程度数据不平衡和跨任务扩展的能力。

(2)提出基于多角度对比学习的实体类型过度依赖消解方法。针对事件要素抽取中存在的实体类型过度依赖,首先定义并研究该依赖在不同事件要素抽取模型架构中的存在性。在此基础上,提出一种基于多角度对比学习的事件要素抽取方法。具体地,构建两个监督对比学习模块,在正样本角度提升具有相同要素类型但具有不同实体类型的实例特征表示间的相似性,而在负样本角度降低具有不同要素类型但具有相同实体类型的实例特征表示间的相似性。进一步,设计循环训练策略提升多角度消解的效率。实验评估表明,提出的方法可以通用有效地消解不同事件要素抽取架构的实体类型过度依赖,并基于预训练语言模型架构取得当前最优性能。

(3)提出基于分离-融合范式的跨事件信息构建与利用方法。首先总结目前通用文本级别事件要素抽取方法存在的内在局限,提出基于多词元链接的建模方法。进一步,为了从不同级别文本中高效利用跨事件依赖信息,提出分离-融合抽取范式,其将跨事件信息获取和事件要素抽取先分开建模再进行融合,从而充分利用不同抽取范式的优势。遵循该范式,提出一种新的多词元链接模型。具体地,引入一个链接模块连接不同事件的触发词和对应的要素类型以提供跨事件信息,同时构建另一个链接模块抽取目标事件的要素信息。此外,提出一个两阶段融合模块将获取的跨事件信息融合到目标事件的要素抽取中。实验结果表明,提出的方法在不同的句子级别和文档级别数据集上均显著超越目前性能最优的事件要素抽取基线。

本文面向事件抽取系统中的数据不平衡依赖、实体类型过度依赖和
跨事件依赖三类共性数据挑战,相应研究分类器自适应知识蒸馏、多角度对比学习和分离-融合范式三种通用性方法,覆盖不同模型架构、不同文本范围和不同事件抽取阶段,以有力提升语言智能化和事件知识图谱构建自动化水平。

\keywords{事件抽取; 共性数据依赖; 数据不平衡依赖; 实体类型过度依赖; 跨事件依赖}
\end{abstract}

\begin{englishabstract}

As a key component in information extraction, event extraction provides structured semantic information that is widely used in diverse natural language scenarios such as fact checking, information retrieval, intelligent question answering, and recommendation systems. In addition, the event knowledge graph based on event extraction also holds broad application prospects. With the rapid development of deep learning technology, existing methods use specific neural network architectures to study different stages of event extraction. Although these methods have made excellent performance progress, there are still the following common data challenges: 1) The data imbalance dependency in different event detection methods limits the ability to identify events and further reduces the overall detection performance; 2) The entity type overdependency in different event argument extraction methods prevents these methods from learning more semantic information besides entity types, which affects the correct predictions of event argument roles; 3) When extending event argument extraction to multi-level text, the inherent limitations and cross-event information restrictions of different event argument extraction methods prevent them from effectively constructing cross-event dependency from text of different levels, which limits the ability of multi-event argument extraction. Therefore, this paper focuses on the common data dependency in event extraction. The main research contents and innovative achievements are as follows:

(1) A data imbalance dependency resolution method based on classifier-adaptation knowledge distillation is proposed. For the data imbalance dependency in event detection, this paper first studies how it affects event detection performance, and defines and introduces sentence-level identification information accordingly to eliminate the performance degradation caused by the data imbalance dependency. On this basis, an event detection method based on classifier-adaptation knowledge distillation is proposed. Specifically, this method introduces the sentence-level identification information transformed according to the training example labels into the event identification enhancement network, and takes it as part of the input to participate in the training. Furthermore, an additional training task is designed to guide the event detection network to automatically learn sentence-level identification information from the original text through the classifier of the event identification enhancement network to improve event detection performance. Experimental evaluation shows that the proposed method can generally and effectively eliminate the performance impact of the data imbalance dependency on different event detection models, and demonstrate the ability to adapt to varying degrees of data imbalance and cross-task expansion.

(2) An entity type overdependency resolution method based on multi-view contrastive learning is proposed. This paper first defines entity type overdependency in event argument extraction and studies its existence in different event argument extraction models. On this basis, an event argument extraction method based on multi-view contrastive learning is proposed. Specifically, two supervised contrastive learning modules are constructed to increase the similarity between the learned feature representations of instances with the same role type but having different entity types from the perspective of positive samples and reduce the similarity between the learned feature representations of instances with different role types but having the same entity type from the perspective of negative samples. Furthermore, a cyclic training strategy is designed to improve the efficiency of multi-view resolution. Experimental evaluation shows that the proposed method can generally and effectively eliminate the entity type overdependency of different event argument extraction architectures and achieve the state-of-the-art performance based on the pre-trained language model.

(3) A cross-event information construction and utilization method based on separation-and-fusion paradigm is proposed. Firstly, the inherent limitations of the current general text-level event argument extraction methods are summarized, and a modelling method based on multiple token linking is proposed. Furthermore, a separation-and-fusion extraction paradigm is proposed to efficiently utilize cross-event dependency information from the text of different levels. It models cross-event information acquisition and event argument extraction separately and then fuses them, thus fully using the advantages of different extraction paradigms. Following this paradigm, a novel multiple token linking model is proposed. Specifically, a linking module is introduced to connect the triggers of different events and the corresponding role types to provide cross-event information, and another linking module is constructed to extract the argument information of the target event. In addition, a two-fold fusion module is proposed to fuse the acquired cross-event information into the argument extraction of the target event. The experimental results show that the proposed method significantly surpasses the state-of-the-art event argument extraction baselines on different sentence-level and document-level datasets.

For the three common data challenges in event extraction systems: data imbalance dependency, entity type overdependency and cross-event dependency, this paper studies three general methods, including classifier-adaptation knowledge distillation, multi-view contrastive learning and separation-and-fusion paradigm, to cover different model structures, different text ranges and different event extraction stages, thus effectively improving language intelligence and event knowledge graph automatic construction.

\englishkeywords{event extraction; common data dependency; data imbalance dependency; entity type overdependency; cross-event dependency}

\end{englishabstract}
